In this first chapter, I will depict the theoretical context of the work. section \ref{ssec:hist_im} will be about histological images, and the different techniques used to prepare the samples to analyze. Histological images recover a fundamental role in medicine and are the pillar of many diagnosis techniques. This discipline borns traditionally from the optical inspection of the tissue slides using a microscope, and it is gradually developing and improving with the advent of computers and digital image processing. It is important tough to understand how the samples are physically prepared, the final target of this work is in fact the virtual reconstruction of this process. In section \ref{ssec:DL} I will introduce the Deep Learning framework and describe how a Neural Network works and actually learns. The most advanced techniques for the automatic image processing implement Deep Learning algorithm, and understanding the general rules behind this discipline is crucial for a good comprehension of this work. In section \ref{ssec:segmentation} I will discuss in particular the problem of image segmentation and how it is tackled with different Neural Network architectures, showing what it is the state of the art of this research field at the moment.
