As mentioned in the introduction, this project wants to produce synthetic histological images paired with their corresponding segmentation mask, to train Neural Networks for the automatization of real histological images analysis. The production of artificial images passes through the processing of a three-dimensional, virtual model of a histological structure, which is the heart of this thesis work. The detailed description of the development of the two proposed histological models will follow the present chapter and will occupy the entire chapter \ref{sec:models}. Here I will dwell, instead, on every technical tool employed during the models' designing phase. From the practical point of view, this project is quite articulated and the development has required the harmonization of many different technologies, tools, and code libraries. The current chapter should be seen as a theoretical complement for chapter \ref{sec:models}, and its reading is suggested to the reader for any theoretical gap or for any further technical deepening. The reader already familiar with those technical tools should freely jump to the models' description.

All the code necessary for the work has been written in a pure \texttt{Python} environment, using several already established libraries and writing by myself the missing code for some specific applications. I decided to code in \texttt{Python} given the thriving variety of available libraries geared toward scientific computation, image processing, data analysis, and last but not least for its ease of use (compared to other programming languages). In each one of the following subsections, I will mention the specific code libraries which have been employed in this project for every technical necessity.
