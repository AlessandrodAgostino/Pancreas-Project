\hl{In this chapter, I will try to resume in the more efficient way my work and the process I followed for the realization of this project.}

In section \ref{ssec:panc_tis_mod}, I will describe all the necessary steps to create the model of a small region of pancreatic tissue, while in section \ref{ssec:derm_tis_mod} I will expose the steps I followed to build a model of dermal tissue. In section \ref{sec:synth_image}, instead, I will show the resulting synthetic images from the sectioning process performed on both the models and all the enrichments and processing necessary to give them the most realistic look I was able to recreate.


The main goal of the present work, as stated before, is to recreate a three-dimensional virtual model of histological tissue as faithfully as possible and then, to perform planar sectioning on it to emulate virtually the traditional histological specimen preparation procedure. The creation of a model of such complex structures is definitely a high-level problem, and it has required a careful designing, made of subsequent stages of improvements. In this work, I will report only two specific attempts of modelization: the first aiming to represent pancreatic tissue, and the second oriented toward dermal tissue.
