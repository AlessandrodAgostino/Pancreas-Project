\chapter*{Conclusions}
\label{chap:concl}
\addcontentsline{toc}{chapter}{\nameref{chap:concl}} %manually adding the unnumbered chapter to toc

In this project I face the problematic of synthetic histological image generation for the purpose of training Neural Networks for the segmentation of real histological images. The manual analysis of histological specimen is a complex, time consuming, and expensive task even if it is a pillar of countless diagnostic techniques. Any form of support for this procedure hence is welcome and endorsed by the health-care system. In particular in this work I focus on the problem of specimens segmentation and its automatization. The most advanced algorithms for image segmentation are based on Deep Learning, and requires the training of extensive and complex neural networks. One of the toughest hurdle to overcome for the training of those NN is the abboundacy and the quantity of pre-labeled example of segmentation on real histological samples. The collection of hundreds of hand-labeled histological samples, with a pixel-level precision is virtually impossible. This work thus proposes a metodology to generate, in a complete automatic way, synthetic pre-labeled histological-like images, that can be used as training material for those NN.

The method I propose consists of the recreation of the traditional histological specimes' preparation, thus is based on the sectioning of a 3D virtual model of a region of histological tissue. The 3D model of a region of a particular type of human tissue is based on physical and physiological considerations, and virtualized. The model is then subject to a virtual sectioning operation, which yields the synthetic sampling of the virtual tissue in which the histological identity of every pixel is perfectly know. This first image will act as segmentation mask for the second, realistic image. In fact, on top of this first image are applied several aesthetical processing and refinments and the final product is the synthetic histological-like image. The pair made of the two images is then perfect for the supervised learning of a NN oriented toward the segmentation of histological images. The production of each pair of images is completely automatic and it does not require the intervention of any human operator, it is thus a scalable process which can produce a great abundance of images. The quality of the images is directly connected to the richness and the quality of the model. The perfect modelization of a region of tissue, let's say human pancreatic tissue, is by far out of reach for this, hence the richness and the fidelity of the produced images is inevitably lower than the real sample. Nevertheless, the quality of the produced images is sufficient to perform the preliminary phase of the training of a NN following a training strategy known as curriculum learning. This learning process consists of giving the NN a copious quantity of lower complexity level example in first instance, reserving the few and sophisticated real hand-labeled histological samples for the finalization of the training.

The first chapter of this thesis is devoted to the contestualization of the present work. It is offered a description of how the histological samples are obtained after a tissue biopsy and how the digitalization process of the images works. The reader is then introduced to the framework of Deep Learing, its fundamental aspect and components. The concept of NN is exposed and it is given the general idea hunderneath the training of a DL-based model. The chapter comes to an end with a general introduction to the segmentation task in computer vision, and how it is currently tackled with state-of-the-art algorithms.

The second chapter collects all the details of every less common technical tool I used during the design of this project. A brief theoretical introduction is proposed for every item besides the thouroug descriprion of its practical use. Some of the arguments touched by this chapter are: quaternions, quasi-random number generation, Voronoi decomposition, style transfer neural networks. In this chapter a section is devoted to the description of a general methodology for computing the 2D section of an arbitray three-dimensional polyhedron. The algorithm here described has been devised and implemented by my self, and then inserted in the workflow of the project. As a conclusion for this chapter I describe the working environment I built for developing this project and the mention of all the code libraries I employeed in my work.

The third chapter is the heart of this work, and it contains the description of all the design choices, and the steps I followed for the development of the two human tissue models I propose: the first of pancreatic tissue and the second of dermal tissue. The development has required the harmonization of many different technical aspects and mathematical tools. The first section is then dedicated to the description of the pancreatic tissue model, which passes through different steps: from a two-dimensional ramification taken as inspiration for the behaviur of blood vessels to the complete three-dimensional model, with its subdivision in labeled cells. The second section is occupied by the description of the dermal tissue model, following the same spirit. The third section instead contains the thorough description of the method to perform the sectioning onto a virtual model and how to process the resulting images. It is necessary to perfom several alternative processing and adjustment to achieve the desired aspect both for the segmentation mask image and for the histologicl-like image.

\hl{The method I propose here for the generation of datasets of synthetic images is self-supporting and ready to work}. By the way, there are many possibilities of improvment and enrichment for the project. One first aspect to strengthen could be the richenss of the models and the collection of available models: adding more elements in the structure and imporving the representation of the cellular level functioning. This would lead to a better quality and representativeness of the synthetic images, that would assist the training of NN in more and different applications. Another aspect that lends itself to improvments is the deveopment of a dedicated style transfer NN targeting the histological texture transfer, which could lead to interesting progresses in the quality of image generation.

There is also the intention to perform an actual attempt of NN training on the images produced with this process. This would be 
